% Gemini theme
% https://github.com/anishathalye/gemini

\documentclass[final]{beamer}


% ====================
% Packages
% ====================

\usepackage[T1]{fontenc}
\usepackage{lmodern}
\usepackage[size=custom,width=84.1,height=118.9,scale=1.0]{beamerposter}
\usetheme{gemini}
\usecolortheme{penn}
\usepackage{graphicx}
\usepackage{booktabs}
\usepackage{tikz}
\usepackage{pgfplots}
\pgfplotsset{compat=1.14}
\usepackage{anyfontsize}
\usepackage{subcaption,mwe}

% ====================
% Lengths
% ====================

% If you have N columns, choose \sepwidth and \colwidth such that
% (N+1)*\sepwidth + N*\colwidth = \paperwidth
\newlength{\sepwidth}
\newlength{\colwidth}
\setlength{\sepwidth}{0.02\paperwidth}
\setlength{\colwidth}{0.47\paperwidth}

\newcommand{\separatorcolumn}{\begin{column}{\sepwidth}\end{column}}

% ====================
% Title
% ====================

\title{Fast Tracking on Heterogeneous Hardware for the ATLAS Event Filter}

\author{Abraham Kahn, on behalf of the ATLAS TDAQ Collaboration}

%\institute[shortinst]{\inst{1} University of Pennsylvania \samelineand \inst{2} College of Everywhere}

% ====================
% Footer (optional)
% ====================

\footercontent{
  %\href{https://www.example.com}{https://www.example.com} \hfill
  CHEP 2023, Norfolk \hfill
  \href{mailto:abkahn@sas.upenn.edu}{abkahn@sas.upenn.edu}
  }
% (can be left out to remove footer)

% ====================
% Logo (optional)
% ====================

% use this to include logos on the left and/or right side of the header:
\logoright{\includegraphics[height=7cm]{images/ATLAS-Logo-Standard.png}}
\logoleft{\includegraphics[height=7cm]{images/EF_Logo.png}}
\logobottomright{\includegraphics[height=2cm]{images/UPenn_logo.png}}

% ====================
% Body
% ====================

\begin{document}

\begin{frame}[t]
\begin{columns}[t]
\separatorcolumn

\begin{column}{\colwidth}

  \begin{block}{Overview}

    The High Luminosity upgrade of the Large Hadron Collider (HL-LHC) will deliver $\mathcal{O}(10)$ times the total integrated luminosity
    of LHC Runs 1-3 combined \cite{TDR}. However, this increased rate of proton-proton collisions per bunch crossing (pile-up) also poses
    significant challenges to the ATLAS Trigger and Data Acquisition System (TDAQ). In order to meet the physics goals and realize the 
    full potential of the HL-LHC, the TDAQ system must be able to quickly execute track and vertex reconstruction in this high pile-up environment.
    The Event Filter (EF) will need to be designed to work at a luminosity of 
    $\mathcal{L} = 7.5 \times 10^{34} \text{ cm}^{-2}\text{s}^{-1}$, with a maximum input rate of 1 MHz and output rate of 10 kHz.

    Industry trends show a shift toward the usage of heterogeneous systems - systems which integrate multiple types of computational units (GPUs, FPGAs, ASICS, etc.)
     - which can achieve higher performance with lower power consumption and lower latency than CPU-only systems. Heterogeneous computing systems prove to be a compelling option 
    for a new EF design, which will be responsible for executing sophisticated track reconstruction algorithms in the ATLAS TDAQ system. 

  \end{block}

  \begin{block}{FPGA-based Tracking}

    One such heterogeneous computing system would use lightweight CPU software to load an event onto an available
    FPGA, which then performs the track reconstruction and outputs a set of track candidates. Each FPGA would 
    have the following implemented in firmware:
    
    \begin{tabular}{cl}  
      \begin{tabular}{c}
        \parbox{0.55\linewidth}{%  change the parbox width as appropiate    
        \begin{itemize}
          \item \textbf{Data Decoding and Clustering:} Provides cluster position and size information from raw 
          front-end ITk data. 
          \item \textbf{Space Point/Stub-Finding:} Pre-filters hit clusters ("stub-finding") and combines strip hit clusters on
          opposite sides of a stave ("space points") to place 3-D constraints on a track location and reduce input to the pattern recognition stage. 
          \item \textbf{Slicing Engine:} To combat large occupancy, data packaged in $\eta-\phi$ regions is further divided in the 
          $z$-direction into "slices". (see below)
          \item \textbf{Pattern Recognition:} The most resource intensive step - identifying track candidates from hit clusters. Two versions 
          of the Hough Transform algorithm have been explored.
          \item \textbf{Duplicate and Fake Removal:} Reject fake tracks using a Neural Network (NN), whose architecture was developed with an FPGA firmware implementation in mind.
          \item \textbf{Coarse Track Fit:} The track candidates from the pattern recognition stage are further cut down using a linearized $\chi^2$ test,
          and their track parameters are evaluated.
        \end{itemize}
        }
        \end{tabular}
        & \begin{tabular}{l}
          \includegraphics[width=0.4\colwidth]{images/Algorithm-Flow.pdf}
      \end{tabular}  \\
    \end{tabular}

  \end{block}

  %There are countless options available for optimizing both of the Hough Transform implementations. To account for non-prompt tracks, an extra
    %"hit extension" can be applied by filling $n$ extra bins in $\phi$ in the accumulator. In addition, extra bins can be
    %added to edges of the accumulator to combat any efficiency loss on region boundaries.

    One of the most important strategies for reducing the occupancy in the accumulator is to further divide an $\eta-\phi$ region 
    into "slices" in the $z$-direction, referred to as $z$-slicing. The main slicing algorithm divides the data into overlapping regions which target tracks 
    passing through a selected key layer. This crucial subdivision is incorporated into both of the Hough Transform implementations.

    \begin{figure}
      \centering
      \includegraphics[width=0.7\colwidth]{images/keylayerslices_r562.png}
      \caption*{\emph{An Illustration of the $z$-slicing strategy. Each colored region represents a slice.}}
    \end{figure}

  \begin{block}{2D Hough Transform for Pattern Recognition}

    In the context of track reconstruction, the Hough Transform converts a hit in position space into a line in momentum space: each hit on 
    the $x-y$ plane is transformed into a line on the $qA/p_\text{T}-\phi_t$ plane, which is the set of possible track $p_\text{T}$ and track $\phi$ 
    values consistent with that hit position:
   
    \begin{align*}
      \frac{qA}{p_\text{T}} = \frac{\sin(\phi_t - \phi_h)}{r_h} \approx \frac{\phi_t - \phi_h}{r_h}
    \end{align*}

    Where $q$ is the charge of the particle, $A$ is the curvature constant for a 2 T Magnetic field, $p_\text{T}$ is the transverse 
    momentum of the particle, and $\phi_t$ is the azimuthal angle of the particle at the origin.

    Therefore, if five lines intersect the same bin in $\frac{qA}{p_\text{T}}-\phi_t$ space, it suggests that those five hits came from the same charged
    particle track, with $p_\text{T}$ and $\phi_t$ equal to the intersection point.


    \begin{figure}[!h]
      \centering
      \begin{subfigure}[t]{.45\textwidth}
          \centering
          \includegraphics[width=\textwidth]{images/HTimage_s2_nopileup.pdf}
          \caption{}
      \end{subfigure}
      \begin{subfigure}[t]{.45\textwidth}
         \centering
         \includegraphics[width=\textwidth]{images/HTimage_s2_pileup.pdf}
         \caption{}
      \end{subfigure}
      \caption*{\emph{Hough transform accumulator for a single muon track without (a) and with (b) pile-up}}
    \end{figure}
  
  \end{block}

\end{column}

\separatorcolumn

\begin{column}{\colwidth}


  \begin{block}{1D Hough Transform for Pattern Recognition}

  An alternative method, the "1D Hough Transform", was also developed to perform track finding. This algorithm makes use of the approximation that all hits in a layer 
  have the same radius, so that the entire set of hits can be represented as one bit string per layer. Then each bit indicates whether a $\phi$ bin contains a hit or not.
  It follows that different $p_\text{T}$ hypotheses simply correspond to shifting the bit strings relative to each other, and a track candidate is found if a $\phi$ bin has a hit in enough
  layers. 
  
  There is a loss in $r$ resolution intrinsic in the assumption that the layers have constant radius, but this is made up for by a much smaller FPGA footprint, which in turn allows
  the use of more detailed $z$ information. The impact parameter $d_0$ can also be incorporated into the $p_\text{T}$ shifts, taking into account tracks 
  which don't originate exactly at the origin.

  \begin{figure}
    \centering
    \includegraphics[width=0.7\colwidth]{images/BitShiftApproach.png}
    \caption*{1D Hough approach. After applying a $p_\text{T}$ shift perform a vertical sum, if it's above a given threshold a track candidate is found.}
  \end{figure}

  \end{block}

  

   

  \begin{block}{Pattern Recognition Performance}

    The truth-matched track candidate efficiency was calculated as a function of $p_\text{T}$ for muons, pions, and electrons in two regions of $\eta$ over 
    the $\phi$ range of $ 0.3 < \phi < 0.5$ \cite{TDR_Amend}. The performance of each Hough implementation is comparable to the Offline fast tracking software in the central part 
    of the ATLAS detector. In the less studied forward region of the detector, the performance is not as good as offline. This may be addressed with further optimization
    of each Hough transform configuration, or potentially by using an alternative pattern recognition algorithm.

    \begin{figure}[!h]
      \centering
      \begin{subfigure}[t]{.3\textwidth}
          \centering
          \includegraphics[width=\textwidth]{images/HoughEff_m_reg_0.pdf}
          \caption*{Muons in $0.1 < \eta < 0.3$}
      \end{subfigure}
      \begin{subfigure}[t]{.3\textwidth}
         \centering
         \includegraphics[width=\textwidth]{images/HoughEff_p_reg_0.pdf}
         \caption*{Pions in $0.1 < \eta < 0.3$}
      \end{subfigure}
      \begin{subfigure}[t]{.3\textwidth}
        \centering
        \includegraphics[width=\textwidth]{images/HoughEff_e_reg_0.pdf}
        \caption*{Electrons in $0.1 < \eta < 0.3$}
     \end{subfigure}
    \end{figure}

    \begin{figure}[!h]
      \centering
      \begin{subfigure}[t]{.3\textwidth}
          \centering
          \includegraphics[width=\textwidth]{images/HoughEff_m_reg_3.pdf}
          \caption*{Muons in $2.0 < \eta < 2.2$}
      \end{subfigure}
      \begin{subfigure}[t]{.3\textwidth}
         \centering
         \includegraphics[width=\textwidth]{images/HoughEff_p_reg_3.pdf}
         \caption*{Pions in $2.0 < \eta < 2.2$}
      \end{subfigure}
      \begin{subfigure}[t]{.3\textwidth}
        \centering
        \includegraphics[width=\textwidth]{images/HoughEff_e_reg_3.pdf}
        \caption*{Electrons in $2.0 < \eta < 2.2$}
     \end{subfigure}
    \end{figure}

  \end{block}

  \begin{block}{Resource Usage}

    The estimated size and total power of an FPGA-based heterogeneous system is shown in the table below \cite{TDR_Amend}.
    The ranges in the number of accelerators and power correspond to the envelope of three different preliminary firmware implementations. 
    
    \begin{table} [htb!]
      \centering
        \begin{tabular}{|c|c|c|}
          \hline
              & Run 4 & Run 5 \\
          \hline
              \# of Accelerator Cards & \multicolumn{2}{c|}{270-680} \\ %%% This is for 200 kHz
              CPU resource requirement [MHS06] & 1.1-1.7 & 1.6-2.4\\ %%% Run 5 is for 200 kHz, case 3, updated rel 21.9 numbers from v1.0 sw report
              \hline \hline
              Accelerator Power [MW] & \multicolumn{2}{c|}{0.08-0.18} \\
              CPU Power [MW] & 0.28-0.42 & 0.32-0.48 \\ \hline %scaled from CPU line using 0.25 W/HS06 for run 4 and 0.20 W/HS06 for run 5
              Total Power [MW] & 0.4-0.6 & 0.4-0.7 \\
              \hline
        \end{tabular}
    \end{table}

      This can be compared to an estimate of the required resources for a CPU-only fast track reconstruction \cite{TDR_Amend}:
      
      \begin{table}[ht!]
        \centering
        \begin{tabular}{|l||c|c||c|c|c|} \hline
        & \multicolumn{2}{c||}{Pile-up 140} & \multicolumn{2}{c|}{Pile-up 200} \\ \hline
        & full-scan & regional & full-scan & regional \\ \hline
      Rate [MHz] & 0.15 & 1.0 & 0.15 & 1.0 \\
      CPU Resource requirement [MHS06] & 3.41 & 2.49 & 5.36 & 3.81 \\ \hline
      Tracking resource requirement [MHS06] & \multicolumn{2}{c||}{5.90} & \multicolumn{2}{c|}{9.17} \\
      Tracking power requirement [MW] & \multicolumn{2}{c||}{1.47} & \multicolumn{2}{c|}{1.83} \\
      \hline
        \end{tabular}
      \end{table}

  \end{block}

  \begin{exampleblock}{Conclusion}

    A heterogeneous commodity system has been demonstrated to be a viable option for EF tracking in ATLAS.
    This has been shown with implementations of the Hough Transform as the pattern recognition algorithms of choice, but 
    a broad range of pattern recognition strategies are still being explored, including using Graph Neural Networks and GPUs.
    The system also lends itself well to integrating the use of Machine Learning algorithms, and there has already been 
    success using an NN for fake rejection.

    An FPGA-based heterogeneous system also offers noticeable power savings in comparison to CPU-only systems, while still meeting 
    the tracking performance requirements of maintaining high efficiency across $\eta - d_0 - z_0 - p_\text{T}$ phase space. In addition to being
    low risk and commercially available, this system is incredibly flexible and can scale or adapt in response to the future physics needs of 
    the ATLAS experiment.

  \end{exampleblock}


  \begin{block}{References}
    %\footnotesize{\bibliographystyle{ieeetr}\bibliography{poster}}
    \footnotesize{
      \begin{thebibliography}{1}

        \bibitem{TDR}
        ATLAS Collaboration, ``{Technical Design Report for the Phase-II Upgrade of the ATLAS
        TDAQ System},'' Tech. Rep. CERN-LHCC-2017-020. ATLAS-TDR-029, CERN, Geneva,
          Sept., 2017. \href{https://cds.cern.ch/record/2285584}{https://cds.cern.ch/record/2285584}
        
        \bibitem{TDR_Amend}
        ATLAS Collaboration, ``{Technical Design Report for the Phase-II Upgrade of the ATLAS Trigger
         and Data Acquisition System - Event Filter Tracking Amendment},'' Tech. Rep. CERN-LHCC-2022-004. 
         ATLAS-TDR-029-ADD-1, CERN, Geneva, Mar. 2022, \href{https://cds.cern.ch/record/2802799}{https://cds.cern.ch/record/2802799
        }
        
        \end{thebibliography}
    }
  \end{block}

\end{column}

\separatorcolumn
\end{columns}
\end{frame}

\end{document}
